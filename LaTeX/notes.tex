% This section sums up important notation, definitions, and key lemmas that are not the main focus of the thesis.
% Reformulating the OSRQuery
% Table -> summary of notations
\chapter{Notations and Preliminaries} 
\label{sec:notes}
In this chapter, we would like to introduce some terms, notations and definitions that are used throughout the thesis, such as the definition for a Sequenced Route Query (SRQ), which we need in order to define the operators.

\textbf{PoIs sets:} We assume that we have $n$ sets $C_1$, $C_2$, ..., $C_n$, which contain points in a 2-dimensional space ${\rm I\!R}^2$ and $dist(., .)$ is a distance function, which obtains the distance between two points in a two dimensional road network. The sets $C_i$ represent the data sets for the different Categories of Interest (CoIs), e.g. restaurants, gas stations etc.. 

\textbf{Category sequence:} $M = (c_1, c_2, ..., c_l)$ is a sequence of categories, if $1 \leq c_i \leq n$ for $1 \leq i \leq l$, where n is the number of present CoI sets $C_i$. The user is only allowed to ask for existing categories.

\textbf{Route:} $R =(r_1, r_2, ..., r_r)$ is a route, if $r_i \in {\rm I\!R}^2$ for each $1 \leq i \leq r$. $R_sp$ is a route that starts from the starting point $sp$: $R_{sp} = (sp, r_1, r_2, ..., r_r)$

\textbf{Route length:} The length of a route $R = (r_1, r_2, ..., r_r)$ is defined as:

\begin{equation}
length(R) = \sum_{i=1}^{r-1} dist(P_i, P_{i+1})
\end{equation}

For $r = 1$ $length(R) = 0$.

\textbf{Sequenced route:} Let $M = (c_1, c_2, ..., c_l)$ be a sequence of CoIs. $R = (r_1, r_2, ..., r_l)$ is a sequenced route that follows the category sequence M, if $P_i \in C_{M_i}$ where $1 \leq i \leq l$. The Points of Interest (PoIs) in the route should belong to the corresponding CoI sets, defined in the category sequence.

\textbf{Optimal sequenced route (OSR) query:} Given a sequence of categories $M = (c_1, c_2, ..., c_l)$ and a starting point $sp$ in ${\rm I\!R}^2$, $Q(sp, M)$ is the Optimal Sequenced Route (OSR) Query, which searches for the shortest (in terms of function $length$) sequenced route $R$ that follows $M$. The length of the complete route is defined as:

\begin{equation}
length(sp, R) = dist(sp, P_1) + length(R)
\end{equation}

All other sequenced routes that follow $M$ are referred to as candidate sequenced routes (SR).

\textbf{Nearest neighbor:} The nearest neighbor from the category point set $C_p$ to a route point $r_q$ is defined as $NN(r_q, c_p)$.

\textbf{Next nearest neighbor:} The next nearest neighbor $r_p'$ from the category point set $C_p$ to a route point $r_q$, relative to the last found  $k$-th neighbor $r_p$ is defined as $KNN(r_q, c_p')$. This is the $k+1$-th neighbor of $r_q$ and the following equation must apply:

\begin{equation}
dist(r_q, r_p') \geq dist(r_q, r_p)
\end{equation} 

Table \ref{table} summarizes all used notations.

\enlargethispage{30pt}

\begin{table}[h!]
\centering
	\begin{tabular}{ |l|p{10cm}| } 
		\hline
		\textit{Symbol} & \textit{Meaning} \\
		\hline
		$C_i$ & a point set for a category in ${\rm I\!R}^2$ \\ 
		\hline
		$|C_i|$ & cardinality of the set $C_i$ \\ 
		\hline
		$n$ & number of point sets $C_i$ \\ 
		\hline
		$dist(., .)$ & distance function in ${\rm I\!R}^2$ \\ 
		\hline
		$M$ & category sequence, $=(c_1, c_2, ..., c_l)$ \\ 
		\hline
		$|M|$ & $l$, size of sequence $M$ = number of items in $M$ \\ 
		\hline
		$c_i$ & $i$th member of $M$ \\ 
		\hline
		$R$ & route, $= (r_1, r_2, ..., r_r)$ \\ 
		\hline
		$|R|$ & $r$, size of route $R$ = number of points in $R$ \\ 
		\hline
		$r_i$ & $i$th point in $R$ \\ 
		\hline
		$length(R)$ & length of $R$ \\ 
		\hline
		$length(sp, R)$ & length of $R_{sp} = (sp, r_1, r_2, ..., r_r)$, $= length(R_{sp})$ \\ 
		\hline
		$Q(sp, M)$ & sequenced route query \\ 
		\hline
		$NN(r_q, c_p)$ & nearest neighbor from the category point set $C_p$ to a route point $r_q$ \\ 
		\hline
		$KNN(r_q, c_p')$ & next nearest neighbor $r_p'$ from the category point set $C_p$ to a route point $r_q$ \\ 
		\hline
	\end{tabular}
\caption{Notations}
\label{table}
\end{table}




