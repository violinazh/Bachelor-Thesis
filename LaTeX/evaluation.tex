\chapter{Experimental studies}
\label{sec:evaluation}

\textbf{Graph model:}
The road network or also known as spatial network is modeled as weighted graph where the crossroads are represented by nodes and roads are represented by the edges connecting the nodes. The weights on the edges in this specific research problem are the distances between the nodes on the edges. The distance between any two points can be found by summing up the lengths of the edges that belong to the shortest path between the two points.

\textbf{Dataset:}
The graph used to conduct the experiments on is constructed using Berlin's spatial datasets from \cite{datasets}, structured in separate CSV files for the crossroads, roads and CoI sets. For the implementation of the operators the datasets are imported into a graph structure of nodes and edges, where each node has a unique id, its latitude and longitude and a list of PoIs that have been mapped to it and each edge has a source and destination node and the distance between the two nodes in kilometers as parameters. Each PoI is mapped to the nearest crossroad and has a unique id, a type, its latitude and longitude and the distance to the node it is mapped to.

The map used for the experiments is the road network of Berlin, with 428769 crossroad nodes, 504229 road edges, 5548 PoIs and 7 category types: restaurant, coffee shop, atms/bank, movie theater, pharmaciy, pubs/bar, gas station (see Table \ref{dataset}). 

\begin{table}[H]
	\centering
	\begin{tabular}{ |c|c|c| } 
		\hline
		\textit{Categories} & \textit{Size} & \textit{Frequency}\\
		\hline
		Restaurants & 2081 & \multirow{3}{3em}{High}\\ 
		Coffee shops & 1002 &\\
		Pubs and bars & 958 &\\  
		\hline
		Atms/Banks & 597 & \multirow{3}{3em}{Middle}\\
		Pharmacies & 589 &\\
		\hline
		Gas stations & 180 & \multirow{3}{3em}{Low}\\
		Movie theaters & 141 &\\ 
		\hline
	\end{tabular}
	\caption{CoIs in Berlin's road network}
	\label{dataset}
\end{table}

\textbf{Technical details:}
The experiments were performed on two Linux machines with AMD Opteron Processor 6212 with 2,60 GHz and Intel Xeon E5-2630 processor with 2.40GHz and respectively 16 CPUs and 128 GB RAM. The experiments for each parameter type were executed on 1000 queries with randomly selected starting points. Finally, the average of the results is reported.

Several experiments were conducted to evaluate the performance of the proposed algorithms. The evaluation criteria, which relate to all the operators, presented in the thesis are the following: (1) processing time (in milliseconds), (2) total number of heap fetches and (3) maximum heap size, representing the work space (WS) of the method. Since most of the route planning services are mobile services, designed to use on the go, response time is of utter importance. In addition to that we want to see how the work space relates to the number of fetch operations, performed on the heap. Furthermore, to highlight the benefits of three of the operators - EO, OR, ORDER, we also run the experiments on the baseline approaches to compare them with the proposed algorithms. 

\section{"Equality" operator}
\label{sec:experimentsEO}

The "equality" operator was evaluated with respect to the effect of following 3 parameters: (1) the query length (cardinality of the category sequence $|M|$), (2) the frequency of the categories $c_i$, $c_j$ in the category sequence $M$ and (3) the distance between the categories $c_i$, $c_j$. Categories $c_i$, $c_j$ represent the CoIs for which equal PoIs should be found.

In the first set of experiments, shown in Figure \ref{fig:eo_length} a), b), c), the "equality" operator was evaluated in terms of query length, which varies from 3 to 7. Queries with length less than 3 would be immediately solved with PNE, which is why we do not consider them in these experiments. 
As we can see in Figure \ref{fig:eo_length} a), the query's processing time increases proportionately to the query's length. This can be explained with the fact, that the search space grows exponentially with increasing length of the category sequence. Figure \ref{fig:eo_length} a) also shows what portion of the total processing time of the proposed approach belongs to executing PNE in the first step of the algorithm. And as expected, both PNE's time and the proposed approach's time increase with the query length. Figure \ref{fig:eo_length} b), c) follow the same trend as a). As the query length increases, the number of heap fetches and the maximum heap size also increase accordingly. The response time for a query with length 7 has not been depicted fully in the graphic due to a lack of space.

\begin{figure}[H]
	\includegraphics[scale=0.33]{images/eo_length_30.png}
	\centering
	\caption{Equality "operator" - experiments on the query length}
	\label{fig:eo_length}
\end{figure}

In the next set of experiments, shown in Figure \ref{fig:eo_frequency} a), b), c), the "equality" operator was evaluated in terms of the frequency of the categories $c_i$, $c_j$. Frequency relates to the size of each CoI dataset (see Table \ref{dataset}) and is categorized into \textit{low}, \textit{middle} and \textit{high}, for a default query length of 5. Figure \ref{fig:eo_frequency} b), c) follow the same trend as the experiments for query length in a). As the frequency of the categories, which are selected to be equal, increases, the processing time, the number of heap fetches and the heap's maximum size increase proportionately. This can be explained with the fact that having more points in the CoI dataset of the equal categories increases the search space of the algorithm and respectively more partial routes are generated, which increases the heap size, number of heap fetches and in turn the processing time of the query.

\begin{figure}[h!]
	\includegraphics[scale=0.33]{images/eo_frequency_30.png}
	\centering
	\caption{"Equality" operator - experiments on the category frequency of the categories $c_i$ and $c_j$}
	\label{fig:eo_frequency}
\end{figure}

In the third set of experiments, shown in Figure \ref{fig:eo_distance} a), b), c), the "equality" operator was evaluated in terms of the  distance between the categories $c_i$, $c_j$ in the category sequence $M$, which varies from 1 to 3, for a query length of 5. When the distance between them is 0, the result is always found with PNE, therefore we do not consider distance 0 in the experiments. As the distance of the categories $c_i$ and $c_j$ increases, the processing time, the number of heap fetches and the heap's maximum size increase proportionately. This stems from the fact that by increasing the distance, the probability that the PoIs at indices $i$ and $j$ would be equal decreases, therefore less of the routes are found in the first step of the algorithm with PNE. This in turn causes the method to continue with the heuristic approach in order to find an optimal route, which increases the processing time, number of heap fetches and also the work space (maximum heap size).

\begin{figure}[h!]
	\includegraphics[scale=0.33]{images/eo_distance_30.png}
	\centering
	\caption{"Equality" operator - experiments on the distance between equal categories $c_i$ and $c_j$}
	\label{fig:eo_distance}
\end{figure}

Finally, the last set of experiments, shown in Figure \ref{fig:eo_distance} a), b), c), compares the baseline approach with the proposed approach in terms of the factor query length. It can be seen that the proposed approach outperforms the baseline approach for all values of the query length. Also with increase in the query length, the values for the processing time, the number of heap fetches and the maximum heap size of the baseline approach increase with a more than a linear rate compared to the the proposed approach. For a query's length of 7 the values for the processing time, the number of heap fetches and the maximum heap size of the baseline approach exceed the graphic's range and are not fully depicted.

In conclusion, the proposed method performs as expected in all studied parameters and outperforms the baseline approach by an exponential amount. 

\begin{figure}[H]
	\includegraphics[scale=0.33]{images/eo2_30.png}
	\centering
	\caption{"Equality" operator - comparison experiments on the proposed approach and the baseline approach in terms of query length}
	\label{fig:eo2_length}
\end{figure}


\section{"Inequality" operator}
\label{sec:experimentsNEO}

The "inequality" operator was also evaluated with respect to the effect of following 3 parameters: (1) query length (cardinality of the category sequence $|M|$), (2) the frequency of the categories $c_i$, $c_j$ and (3) the distance between the categories $c_i$, $c_j$ in the category sequence $M$. Categories $c_i$, $c_j$ represent the CoIs for which unequal PoIs should be found.

In the first set of experiments, shown in Figure \ref{fig:neo_length} a), b), c), the "inequality" operator was evaluated in terms of query length, which varies from 3 to 7. As we can see in Figure \ref{fig:neo_length} a), the query processing time increases proportionately to the query length, as was the case with the "equality" operator.  Figure \ref{fig:neo_length} b), c) follow the same trend of a). As the query length increases, the number of heap fetches and the maximum heap size also increase. Similar to the "equality" operator, the response time and the number of heap fetches for a query with length 7 have not been depicted fully in the graphic due to a lack of space, but it can be seen that they increase exponentially with the increase in query length.\newline

\begin{figure}[H]
	\includegraphics[scale=0.33]{images/neo_length_30.png}
	\centering
	\caption{"Inequality" operator - experiments on the query length}
	\label{fig:neo_length}
\end{figure}

\enlargethispage{\baselineskip}

In the second set of experiments, shown in Figure \ref{fig:neo_frequency} a), b), c), the "inequality" operator was evaluated in terms of the frequency of the categories $c_i$, $c_j$, which can be \textit{low}, \textit{middle} and \textit{high}, for a default query length of 5.  
Here we can see more interesting results compared to the "equality" operator. The \textit{low} frequency has higher results on all parameters than the \textit{middle} frequency. This can be explained with the fact that when the category which has to produce equal PoIs is less frequent, then the possibility of PNE reaching a route with unequal PoIs is lower. In this case the search space of unequal PoIs is larger, compared to when the category of the unequal PoIs has a middle frequency. And when the category's frequency is high, the algorithm must inspect more PoIs in the case that a route with unequal PoIs is not immediately found with PNE. Therefore the algorithm generates more routes and the time increases exponentially. 
The results in \ref{fig:neo_frequency} b), c) are equivalent to the results in \ref{fig:neo_frequency} a). 

\pagebreak

\begin{figure}[H]
	\includegraphics[scale=0.33]{images/neo_frequency_30.png}
	\centering
	\caption{"Inequality" operator - experiments on the category frequency of the categories $c_i$ and $c_j$}
	\label{fig:neo_frequency}
\end{figure}

In the third set of experiments, shown in Figure \ref{fig:neo_distance} a), b), c), the "inequality" operator was evaluated in terms of the  distance between the categories $c_i$, $c_j$ in the category sequence $M$, which varies from 0 to 3, for a default query length of 5.  
Here we can also see more interesting results compared to the "equality" operator. When the distance between the mentioned categories is 0, the result, usually found with PNE, always contains equal points at indices $i$ and $j$. Therefore the algorithm for the "inequality" operator has to inspect more points in order to find the optimal route, in which the PoIs at indices $i$ and $j$ are not equal to each other. This increases the search space and in turn the number of heap fetches, the maximum heap size and the processing time increase as well. For distances 1, 2, 3 no obvious argumentation can be applied, because here we are not able to judge the possibility for equal PoIs objectively.
% Is this argumentation okay?
Nevertheless, all three performance parameters - processing time, number of heap fetches and maximum heap size, follow the same trend and increase proportionately to each other.

In conclusion, our proposed approach performs as expected and for a medium query length (5) delivers fast results in a response time of a couple milliseconds.

\begin{figure}[H]
	\includegraphics[scale=0.33]{images/neo_distance_30.png}
	\centering
	\caption{"Inequality" operator - experiments on the distance between equal categories $c_i$ and $c_j$}
	\label{fig:neo_distance}
\end{figure}


\section{"Or" operator}
\label{sec:experimentsOr}

The "or" operator was studied with respect to the type and number of operands, in a default query length of 5. Three different types of queries were issued, in which we changed the type and number of or operands in the first OR sequence $OR_1$ of the query, while the other four OR sequences only contained one category sequence with a single category. For \textit{2 simple operands} the first OR sequence contained two category sequences with one single category each, for \textit{3 simple operands} the first OR sequence contained three category sequences with one single category each and for \textit{2 complex operands} the first OR sequence contained two complex category sequences with two categories in each. In this way the complexity of the problem was gradually increased to see how the baseline approach compares to proposed approach of the algorithm.  
Figure \ref{fig:or} a), b), c) shows that the processing time, the number of heap fetches and the maximum heap size increase proportionately with the complexity of the query. As seen from the experiments the proposed approach is also with more than a linear rate faster and more efficient in terms of the heap size, as the complexity of the query increases. 

\begin{figure}[H]
	\includegraphics[scale=0.29]{images/or_30.png}
	\centering
	\caption{"Or" operator - experiments on the type and number of operands}
	\label{fig:or}
\end{figure}


\section{"Order" operator}
\label{sec:experimentsOrder}

The "order" operator was evaluated with respect to the number of fixed positions in a default query with length of 5. The number ranges between 0 and 3. When the number of fixed positions is 4 or 5, the problem can be solved with PNE, therefore we do not consider these numbers in the experiments. The complexity of the problem is gradually decreased to see how the baseline approach compares to the proposed algorithm.  
Figure \ref{fig:order} a), b), c) shows that the processing time, the number of heap fetches and the maximum heap size decrease proportionately with the complexity of the query. Compared with the baseline approach, the proposed method is with an exponential rate faster and also more efficient in terms of heap size, especially for unordered queries with 0 fixed positions. The values for the processing time, the number of heap fetches and the maximum heap size of the baseline approach exceed the graphic's range for 0 fixed positions and are not fully depicted. 

\begin{figure}[H]
	\includegraphics[scale=0.33]{images/order_30.png}
	\centering
	\caption{"Order" operator - experiments on the number of fixed positions}
	\label{fig:order}
\end{figure}

\section{Evaluation}
\label{sec:eval}

We can conclude that we have successfully developed four operators for the route query language, which perform as expected in all quantitative criteria such as time, heap fetch operations and required work space. They also outperform the baseline methods with more than a linear rate with increasing of the search space and therefore prove to be effective solutions.

It is important to mention that the implemented approaches could easily be modified to return \textit{k} number of optimal routes, if the user may be interested in this. This could be achieved by simply keeping the generated candidate routes separately. 
Also, the proposed approaches are not only limited to using the route length  as a quantifying measure. Alternative to this could be the duration of the routes. In different scenarios, the methods could also be extended to consider PoI ratings and subjective preferences such as preferred length of the route or traveling time to specific location.  

% We do not report I/O cost separately, because for all methods in all settings, the I/O time is no more than a few milliseconds; i.e., at least an order of magnitude lower than the total response time. 