\SetKw{Return}{return}
\SetKw{Break}{break}

\SetKwFunction{PNE}{PNE}
\SetKwFunction{NN}{NN}
\SetKwFunction{KNN}{kNN}
\SetKwFunction{modifyRouteA}{modifyRouteA}
\SetKwFunction{modifyRouteB}{modifyRouteB}

\SetKwBlock{A}{a)}{end}
\SetKwBlock{B}{b)}{end}

\section{"Or" operator}

\subsection{Motivation}
\label{sec:motOR}
The "or" operator gives the user flexibility to express his need for different route alternatives. It represents the disjunction operator, usually present in query languages. In the context of a sequenced route query, the user can specify multiple categories or a list of categories, which can be disjoint. Then the query is responsible for finding the best route out of all the options that the user has specified. The best route is qualified based on length.

Suppose that a user poses the following query: first he has time to go to either a bank or a pharmacy, then he has a date in a movie theater and after that he plans to go to the restaurant. The proposed approach uses a modified PNE to generate an optimal result for this type of query.

\subsection{Problem definition} 
\label{sec:problemOr}
In order to explain the or query, first we need to define what an OR sequence is: 

\textbf{OR sequence:} An OR sequence $OR = (M_1, M_2, ..., M_m)$ represents the disjunction of category sequences, such as $M_1 = (c_1, c_2, ..., c_l)$. At least one of the category sequences, defined in this OR sequence, must be present in the final result of the query.

The or query is defined as follows:

\textbf{Or query:} Given a sequence of OR sequences $S = (OR_1, OR_2, ..., OR_n)$ and a starting point $sp$ in ${\rm I\!R}^2$
$Q(sp, S)$ is a Sequenced Route (SR) Query, which searches for the shortest (in terms of function $length$) sequenced route $R$ that follows one of the possible permutations of the sequence $S$. 
For example $P = (M_a, M_b, ..., M_z)$ is a permutation of $S$, where $M_a = (c_{a1}, c_{a2}, ..., c_{al})$, $M_b = (c_{b1}, c_{b2}, ..., c_{bl})$ and $M_z = (c_{z1}, c_{z2}, ..., c_{zl})$ build a category sequence \newline
$M = (c_{a1}, c_{a2}, ..., c_{al}, c_{b1}, c_{b2}, ..., c_{bl}, c_{z1}, c_{z2}, ..., c_{zl})$.

%\subsection{Precomputations} 
%\label{sec:precompOr}
%The precomputations are the same as for the or operator in Section \ref{sec:precompEO}.

% Introducing the proposed algorithm + proving the correctness
\subsection{Proposed approach} 
\label{sec:approachOr}
The "or" operator is designed using the PNE approach, proposed in \cite{OSR}. It progressively inspects each option $M_i$ from the OR sequences $OR_i$ in \newline $S = (OR_1, OR_2, ..., OR_n)$, compares them and continues with the best one, based on length, until it reaches a full sequenced route.

% Explaining the algorithm step by step
\subsubsection{Algorithm}
\label{sec:algortihmOr}
Algorithm \texttt{\nameref{alg:or}} starts by initializing the $heap$, ordered by the length of the routes, and the first PSR (line 1 to 16). It then proceeds to inspect and modify the routes on the $heap$ based on their length, until a SR is reached. As can be seen in line 2, a PSR is build with each $M_i$ from the first OR sequences $OR_1$ in $S = (OR_1, OR_2, ..., OR_n)$. The algorithm also checks for each $M_i$ (line 5) if it contains a single category or a sequence of categories. In the first case, it finds the nearest neighbor in $C_{M_i}$ to the starting point as the PNE algorithm would, otherwise it finds the nearest neighbor in $C_{M_i[1]}$ to the starting point and initializes the list of categories $PSR.categories$ for the $PSR$. From all these generated PSR the shortest one is chosen and put on the $heap$.

The the algorithm proceeds with examining the routes on the $heap$ by $index(PSR)$ and modifying them according to PNE. $index(PSR)$ indicates the index of the last found OR sequence $OR_s$ in $S = (OR_1, OR_2, ..., OR_n)$ for the fetched $PSR$. When the current route on the $heap$ is a full SR, where $s = n$, indicating the last OR sequence $OR_n$, then the optimal route has been found (line 21 to 24). Otherwise the algorithm modifies the current PSR according to cases a) and b) as in the PNE algorithm (line 25 to 30). \newline

\begin{algorithm}[H]
\caption{OrOperator}
\label{alg:or}
	
	\SetKwInOut{Input}{Input}\SetKwInOut{Output}{Output}
	
	\Input{$(sp, S = (OR_1, OR_2, ..., OR_n))$}
	\Output{$R = (r_1, r_2, ..., r_l)$}
	\BlankLine
	
	$initialize$ $heap$\; 
	
	\ForEach(Finding all the possible neighbors to the starting point){$M_i$ in $OR_1$}{
		$initialize$ $shortestPSR$\;
		\tcp{Building a new $PSR$ with $M_i$}
		\eIf{$|M_i| = 1$}{
			\tcp{$M_i$ only contains one category}
			\NN{$sp, C_{M_{i}}$}\;
			$update$ $shortestPSR$\;
		}
		{
			\tcp{$M_i$ is a sequence of categories}
			\tcp{The neighbor from the first category $M_i[1]$ is found and the rest of the categries from $M_i$ are put into $categories$ for the specific PSR}
			\NN{$sp, C_{M_i[1]}$}\;
			$PSR.categories \leftarrow M_i[2..l]$\;
			$update$ $shortestPSR$\;
		}	
		
	}
	$firstPSR \leftarrow shortestPSR$\;
	place $firstPSR$ on $heap$\;
	
	\BlankLine
	
	\tcp{At each iteration of PNE a $PSR$ (partial sequenced route) is fetched and examined based on its length and it is checked}
	
	fetch a $PSR$ from the $heap$\;
	\tcp{$index(PSR)$ indicates the index of the last found OR sequence $OR_s$ in $S = (OR_1, OR_2, ..., OR_n)$ for the fetched $PSR$. It is a full SR, when $s = n$, which indicates the last OR sequence $OR_n.$}
	\Switch{$s = index(PSR)$}{
		\Case{$s == n$}{
			$PSR$ is the optimal route\;
			\Return $PSR$\;
		}
		\Case{$s \neq n$}{
			\tcp{a)}
			\modifyRouteA{$PSR$}\;
			
			\tcp{b)}
			\modifyRouteB{$PSR$}\;
		}
	}

\end{algorithm}

In Procedure \texttt{\ref{proc:modifyRouteA}} it is differentiated between a complete PSR and an incomplete PSR. 
A complete PSR has an empty list of $PSR.categories$, which means that the next OR sequence can be examined (line 1 to 19). A PSR is build with each $M_i$ from the OR sequences $OR_s$ in $S = (OR_1, OR_2, ..., OR_n)$. The algorithm also checks for each $M_i$ (line 6) if it contains a single category or a sequence of categories. In the first case, it finds the nearest neighbor to $r_{|PSR|}$ in $C_{M_{i}}$ as the PNE algorithm would (line 6 to 10), otherwise it finds the nearest neighbor to $r_{|PSR|}$ in $C_{M_i[1]}$ and initializes the list of categories $PSR.categories$ for the $PSR$ (line 10 to 16). From these generated PSRs the shortest one is chosen and put on the $heap$.
For an incomplete PSR the next nearest neighbor to $r_{|PSR|}$ in $C_{PSR.categories[1]}$ is found (line 20 to 25). The indicator $PSR.prevPosition$ for the PSR is also updated to indicate that in \texttt{\ref{proc:modifyRouteB}} the k-th neighbor to the second to last PoI should be found.
Trimming is also performed (see \texttt{\ref{proc:trim}}).

In Procedure \texttt{\ref{proc:modifyRouteB}} it is checked if the indicator $PSR.prevPosition$ for the PSR is set ot $true$ (line 1). If that is the case, it means, that the OR sequence of the previous position $OR_{s-1}$ should be examined again (line 1 to 19). A PSR is build with each $M_i$ from the OR sequences $OR_{s-1}$ in $S = (OR_1, OR_2, ..., OR_n)$. The algorithm also checks for each $M_i$ (line 5) if it contains a single category or a sequence of categories. In the first case, it finds the k-th nearest neighbor $r_{|PSR|-1}$ in $C_{M_{i}}$ as the PNE algorithm would (line 6), otherwise it finds the k-th nearest neighbor to $r_{|PSR|-1}$ in $C_{M_i[1]}$ and initializes the list of categories $PSR.categories$ for the $PSR$ (line 10 to 16). From these generated PSRs the shortest one is chosen and put on the $heap$. 
If the indicator $PSR.prevPosition$ for the PSR is set ot $false$  it means that the k-th nearest neighbor to $r_{|PSR|}$ in $C_{PSR.categories[1]}$ is found (line 20 to 25). \newline

%\enlargethispage{\baselineskip}

\begin{procedure}[H]
	\caption{modifyRouteA-or($PSR$)}
	\label{proc:modifyRouteA}
	
	\eIf{$PSR.categories$ is empty}{
		\tcp{We continue with next OR sequence}
		\ForEach(){$M_i$ in $OR_s$}{
			$initialize$ $shortestPSR$\;
			\tcp{Building a new $PSR$ with $M_i$}
			\eIf{$|M_i| = 1$}{
				\tcp{$M_i$ only contains one category}
				\NN{$r_{|PSR|}, C_{M_{i}}$}\;
				$update$ $shortestPSR$\;
			}
			{
				\tcp{$M_i$ is a sequence of categories}
				\tcp{The neighbor from the first category $M_i[1]$ is found and the rest of the categries from $M_i$ are put into $categories$ for the specific PSR}
				\NN{$r_{|PSR|}, C_{M_i[1]}$}\;
				$PSR.categories \leftarrow M_i[2..l]$\;
				$update$ $shortestPSR$\;
			}	
		}
		$newPSR \leftarrow shortestPSR$\;
		
	}{
		\tcp{We continue to find the PoI of the next category in the $PSR.categories$}
		\NN{$r_{|PSR|}, C_{PSR.categories[1]}$}\;
		$remove$ $M_i[1]$ from $PSR.categories$\;
		\tcp{$PSR.prevPosition$, when set to $true$ indicates that in \modifyRouteB{$PSR$} all the sequences in the OR sequence of the previous position $OR_s$ should be traversed and checked again, otherwise the k-th neigbor to the second to last PoI is found}
		$PSR.prevPosition \leftarrow false$\;
		$update$ $newPSR$\;
	}
	
	\tcp{Trimming is also done (see \ref{proc:trim})}
	place $newPSR$ on $heap$\;
\end{procedure}

\pagebreak

\begin{procedure}[H]
\caption{modifyRouteB-or($PSR$)}
\label{proc:modifyRouteB}
	
	\eIf{$PSR.prevPosition$}{
		\tcp{We check the OR sequence of the previous position $OR_{s-1}$}
		\ForEach(){$M_i$ in $OR_{s-1}$}{
			$initialize$ $shortestPSR$\;
			\tcp{Building a new $PSR$ with $M_i$}
			\eIf{$|M_i| = 1$}{
				\tcp{$M_i$ only contains one category}
				\KNN{$r_{|PSR|-1}, C_{M_{i}}$}\;
				$update$ $shortestPSR$\;
			}
			{
				\tcp{$M_i$ is a sequence of categories}
				\tcp{The neighbor from the first category $M_i[1]$ is found and the rest of the categries from $M_i$ are put into $categories$ for the specific PSR}
				\KNN{$r_{|PSR|-1}, C_{M_i[1]}$}\;
				$PSR.categories \leftarrow M_i[2..l]$\;
				$update$ $shortestPSR$\;
			}	
			
		}
		$newPSR \leftarrow shortestPSR$\;
	}{
		\tcp{We find the k-th neighbor to the second to last PoI}
		\KNN{$r_{|PSR|-1}, C_{M_{|PSR|}}$}\;
		$update$ $newPSR$\;
	}
	
	
	place $newPSR$ on $heap$\;
\end{procedure}

\subsubsection{Running example}
We describe the algorithm for the "or" operator using the example in Section \ref{sec:motOR}, using the road network in Figure \ref{fig:example}. The user wanted to visit either a bank or a pharmacy, then go to the movies and then to a restaurant (i.e., $S = (OR_1, OR_2, OR_3)$, $|S| = n = 3$, $OR_1 = ((b), (ph))$, $OR_2 = ((mt))$, $OR_2 = ((r))$). In Figure \ref{heapOR} the partial routes stored in the $heap$ in each step of the algorithm are displayed.
First, all routes from the starting point $sp$ to all possible category sequences in OR sequence $OR_1$ $(b)$ and $(ph)$ are generated. Since there is only one category the simple case is executed, where the nearest neighbor is found. Routes $(b_1)$ with length 2 and $(ph_1)$ with length 3 are generated and compared and the second one is chosen, based on its length, and placed on the $heap$. In step 2 route $(b_1)$ fetched and the nearest neighbor from the next OR sequence $OR_2$ is found. $OR_2$ only contains one category sequence with one category, therefore this is a simple case $modifyRouteA$, where the nearest neighbor is found. In $modifyRouteB$ the next nearest neighbor to the starting point from the OR sequence $OR_1$ is found, which is $ph_1$ with length 3, because $b_2$ with length 4 is further apart. 
In step 6 a full route $(b_1, mt_1, r_2)$ with length 9 is generated and placed on the $heap$ as the candidate route. Then in step 7 another SR $(ph_1, mt_1, r_2)$ with length 10 is generated, but it is longer than the candidate route, previously generated, so it gets discarded. Finally, the optimal route $(b_1, mt_1, r_2)$ with length 9 is returned.

\begin{table}[h]
	\centering
	\begin{tabular}{ |l|l| } 
		\hline
		Step & Heap contents (PSR $R : length(r), index(R)$) \\
		\hline
		1 & $(b_1 : 2, 1)$ \\ 
		\hline
		2 & $(ph_1 : 3, 1), (b_1, mt_1 : 5, 2)$ \\ 
		\hline
		3 & $(ph_2 : 3, 1), (ph_1, mt_1 : 6, 2), (b_1, mt_1 : 5, 2)$ \\ 
		\hline
		4 & $(b_2 : 4, 1), (b_1, mt_1 : 5, 2), (ph_1, mt_1 : 6, 2), (ph_2, mt_1 : 7, 2)$ \\ 
		\hline
		5 & $(b_1, mt_1 : 5, 2), (ph_1, mt_1 : 6, 2), (ph_2, mt_1 : 7, 2), (b_2, mt_1 : 9, 2)$ \\ 
		\hline
		6 & $(ph_1, mt_1 : 6, 2), (ph_2, mt_1 : 7, 2), (b_1, mt_1, r_2 : 9, 3), (b_2, mt_1 : 9, 2)$ \\ 
		\hline
		7 & $(ph_2, mt_1 : 7, 2), (b_1, mt_1, r_2 : 9, 3), (b_2, mt_1 : 9, 2),$ \st{$(ph_1, mt_1, r_2 : 10, 3)$} \\ 
		\hline
		8 & $(b_1, mt_1, r_2 : 9, 3), (b_2, mt_1 : 9, 2),$ \st{$(ph_2, mt_1, r_2 : 11, 3)$} \\ 
		\hline
	\end{tabular}
	\caption{Steps of the OR algorithm using the road network from Figure \ref{fig:example}}
	\label{heapOR}
\end{table}

% Proving the correctness and comparing to the baseline/trivial approach
\subsubsection{Correctness}
\label{sec:correctnessOr}
We need to prove that no suitable partial sequenced routes have been pruned by the proposed algorithm.

\textbf{Lemma 2:} For given query $Q(sp, S)$, the algorithm for the "or" operator returns an optimal route in terms of route length.

\textit{Proof by contradiction:} Let $R' = (r_1', r_2', ..., r_l')$ to be the optimal route and $R = (r_1, r_2, ..., r_l)$ to be the route that the algorithm finds. We assume $length(R') < length(R)$ and show that this is not possible. In order to do this, we differentiate between two separate cases: 

Case 1: At some point in time we have the partial routes $P = (r_1, ..., r_k, r_{k+1})$ and $P' = (r_1, ..., r_k, r'_{k+1})$, which differ only in their last PoI. Since the result route is $R$ at this point the algorithm must choose $P$ as the partial route to extend. This means that $P$ has a shorter length than $P'$, because the heap is ordered by the length of the route. Since the algorithm returns R as the optimal route it means that $length(P') > length(R)$. But because it also holds true that $length(R') > length(P')$, it implies $length(R') > length(R)$, which is a contradiction.

Case 2: At no point partial route $P'$ exists, which is to be extended to sequenced route $R'$. This means that the length of the partial route would exceed the length of the candidate route on top of the heap $length(P') > length(R)$ and is therefore never generated. This implies $length(R') > length(R)$ and this is a contradiction.

Lemma 2 shows that the algorithm for the "or" operator progressively searches for candidate sequenced routes, which follow the possible combinations of category sequences, and returns the optimal route. Thus the correctness of the proposed approach is proved.

\subsection{Baseline approach} 
\label{sec:baselineOr}
A naive solution to find an optimal route with the "or" operator would be to run the PNE algorithm on all possible permutations of the query and to find the shortest route out of all. This is known as the baseline approach.

\subsubsection{Correctness}
The correctness of the baseline algorithm is implied by the correctness of PNE itself.