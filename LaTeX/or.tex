\SetKw{Return}{return}
\SetKw{Break}{break}

\SetKwFunction{PNE}{PNE}
\SetKwFunction{NN}{nearestNeigbour}
\SetKwFunction{KNN}{kNearestNeigbour}
\SetKwFunction{modifyRouteA}{modifyRouteA}
\SetKwFunction{modifyRouteB}{modifyRouteB}

\section{Or operator}
The or operator gives the user flexibility to express his need for different route alternatives. It represents the disjunction operator, usually present in query languages. In the context of a sequenced route query, the user can specify multiple categories or a list of categories, which can be disjoint. Then the query is responsible for finding the best route out of all the options that the user has specified. The best route is qualified based on length.

\subsection{Problem definition} 
\label{sec:problemOr}
In order to explain the or query, first we need to define what an OR sequence is:\newline
\textbf{OR sequence:} An or sequence $OR = (M_1, M_2, ..., M_m)$ represents the disjunction of category sequences, such as $M_1 = (c_1, c_2, ..., c_l)$. At least one of the category sequences, defined in this or sequence, must be present in the final result of the query.\newline

The or query is defined as follows: \newline
\textbf{Or query:} Given a sequence of OR sequences $S = (OR_1, OR_2, ..., OR_n)$ and a starting point $sp$ in ${\rm I\!R}^2$
$Q(sp, S)$ is a Sequenced Route (SR) Query, which searches for the shortest (in terms of function $length$) sequenced route $R$ that follows one of the possible permutations of the sequence $S$. 
For example $P = (M_a, M_b, ..., M_z)$ is a permutation of $S$, where $M_a = (c_{a1}, c_{a2}, ..., c_{al})$, $M_b = (c_{b1}, c_{b2}, ..., c_{bl})$ and $M_z = (c_{z1}, c_{z2}, ..., c_{zl})$ build a category sequence \newline
$M = (c_{a1}, c_{a2}, ..., c_{al}, c_{b1}, c_{b2}, ..., c_{bl}, c_{z1}, c_{z2}, ..., c_{zl})$.

%\subsection{Precomputations} 
%\label{sec:precompOr}
%The precomputations are the same as for the or operator in Section \ref{sec:precompEO}.

% Introducing the proposed algorithm + proving the correctness
\subsection{Proposed approach} 
\label{sec:approachOr}
The or operator is designed using the PNE approach, proposed in \cite{OSR}. It progressively inspects each option $M_i$ from the or sequences $OR_i$ in $S = (OR_1, OR_2, ..., OR_n)$, compares them and continues with the best one, based on length, until it reaches a full sequenced route.

% Explaining the algorithm step by step
\subsubsection{Algorithm}
\label{sec:algortihmOr}
The algorithm (shown in \ref{alg:or}) starts by initializing the heap, ordered by the length of the routes, and the first PSR (line 1 to 16). It then proceeds to inspect and modify the routes on the heap based on their length, until a SR is reached. As can be seen in line 2, a PSR is build with each $M_i$ from the first or sequences $OR_1$ in $S = (OR_1, OR_2, ..., OR_n)$. The algorithm also checks for each $M_i$ (line 5) if it contains a single category or a sequence of categories. In the first case, it finds the nearest neighbor in $U_{M_i}$ to the starting point as the PNE algorithm would, otherwise it finds the nearest neighbor in $U_{M_i[1]}$ to the starting point and initializes the list of categories $PSR.categories$ for the $PSR$. From all these generated PSR the shortest one is chosen and put on the heap.

Modified PNE proceeds with examining the routes on the heap by $index(PSR)$ and modifying them according to PNE. $index(PSR)$ indicates the index of the last found or sequence $OR_s$ in $S = (OR_1, OR_2, ..., OR_n)$ for the fetched $PSR$. It is a full SR, . When the current route on the heap is a full SR, where $s = n$, which indicates the last or sequence $OR_n.$, then the optimal route has been found (line 21 to 24). Otherwise the algorithm modifies the current PSR according to cases a) and b) as in the PNE algorithm \ref{alg:PNE} (line 25 to 30). \newline

\begin{algorithm}[H]
	\label{alg:or}
	\caption{orOperator}
	\SetKwInOut{Input}{Input}\SetKwInOut{Output}{Output}
	
	\Input{$(sp, S = (OR_1, OR_2, ..., OR_n))$}
	\Output{$R = (r_1, r_2, ..., r_l)$}
	\BlankLine
	
	$initialize$ $heap$\; 
	
	\ForEach(Finding all the possible neighbors to the starting point){$M_i$ in $OR_1$}{
		$initialize$ $shortestPSR$\;
		\tcp{Building a new $PSR$ with $M_i$}
		\eIf{$|M_i| = 1$}{
			\tcp{$M_i$ only contains one category}
			\NN{$sp, U_{M_{i}}$}\;
			$update$ $shortestPSR$\;
		}
		{
			\tcp{$M_i$ is a sequence of categories}
			\tcp{The neighbor from the first category $M_i[1]$ is found and the rest of the categries from $M_i$ are put into $categories$ for the specific PSR}
			\NN{$sp, U_{M_i[1]}$}\;
			$PSR.categories \leftarrow M_i[2..l]$\;
			$update$ $shortestPSR$\;
		}	
		
	}
	$firstPSR \leftarrow shortestPSR$\;
	place $firstPSR$ on heap\;
	
	\BlankLine
	
	\tcp{At each iteration of PNE a $PSR$ (partial sequenced route) is fetched and examined based on its length and it is checked}
	
	fetch a $PSR$ from the $heap$\;
	\tcp{$index(PSR)$ indicates the index of the last found or sequence $OR_s$ in $S = (OR_1, OR_2, ..., OR_n)$ for the fetched $PSR$. It is a full SR, when $s = n$, which indicates the last or sequence $OR_n.$}
	\Switch{$s = index(PSR)$}{
		\Case{$s == n$}{
			$PSR$ is the optimal route\;
			\Return $PSR$\;
		}
		\Case{$s \neq n$}{
			\tcp{a)}
			\modifyRouteA{$PSR$}\;
			
			\tcp{b)}
			\modifyRouteB{$PSR$}\;
		}
	}

\end{algorithm}

In case a) \ref{proc:modifyRouteA} it is differentiated between a complete PSR and incomplete PSR. 
A complete PSR has an empty list of $PSR.categories$, which means that the next OR sequence can be examined (line 1 to 19). A PSR is build with each $M_i$ from the or sequences $OR_s$ in $S = (OR_1, OR_2, ..., OR_n)$. The algorithm also checks for each $M_i$ (line 6) if it contains a single category or a sequence of categories. In the first case, it finds the nearest neighbor to $r_{|PSR|}$ in $U_{M_{i}}$ as the PNE algorithm would (line 6 to 10), otherwise it finds the nearest neighbor to $r_{|PSR|}$ in $U_{M_i[1]}$ and initializes the list of categories $PSR.categories$ for the $PSR$ (line 10 to 16). From all these PSR the shortest one is chosen and put on the heap.
For an incomplete PSR the next nearest neighbor to $r_{|PSR|}$ in $U_{PSR.categories[1]}$ is found (line 20 to 25). The indicator $PSR.prevPosition$ for the PSR is also updated to indicate that in \modifyRouteB{$PSR$} the k-th neighbor to the second to last PoI should be found.
Trimming is also performed (see \ref{proc:trim}). \newline

\begin{procedure}[H]
	\label{proc:modifyRouteA}
	\caption{modifyRouteA($PSR$)}
	\eIf{$PSR.categories$ is empty}{
		\tcp{We continue with next OR sequence}
		\ForEach(){$M_i$ in $OR_s$}{
			$initialize$ $shortestPSR$\;
			\tcp{Building a new $PSR$ with $M_i$}
			\eIf{$|M_i| = 1$}{
				\tcp{$M_i$ only contains one category}
				\NN{$r_{|PSR|}, U_{M_{i}}$}\;
				$update$ $shortestPSR$\;
			}
			{
				\tcp{$M_i$ is a sequence of categories}
				\tcp{The neighbor from the first category $M_i[1]$ is found and the rest of the categries from $M_i$ are put into $categories$ for the specific PSR}
				\NN{$r_{|PSR|}, u_{M_i[1]}$}\;
				$PSR.categories \leftarrow M_i[2..l]$\;
				$update$ $shortestPSR$\;
			}	
		}
		$newPSR \leftarrow shortestPSR$\;
		
	}{
		\tcp{We continue to find the PoI of the next category in the $PSR.categories$}
		\NN{$r_{|PSR|}, U_{PSR.categories[1]}$}\;
		$remove$ $M_i[1]$ from $PSR.categories$\;
		\tcp{$PSR.prevPosition$, when set to $true$ indicates that in \modifyRouteB{$PSR$} all the sequences in the OR sequence of the previous position $OR_s$ should be traversed and checked again, otherwise the k-th neigbor to the second to last PoI is found}
		$PSR.prevPosition \leftarrow false$\;
		$update$ $newPSR$\;
	}
		
	\tcp{Trimming is also done (see \ref{proc:trim})}
	place $newPSR$ on heap\;
\end{procedure}

\pagebreak


In case b) \ref{proc:modifyRouteB} it is checked if the indicator $PSR.prevPosition$ for the PSR is set ot $true$ (line 1). If that is the case, it means, that the OR sequence of the previous position $OR_{s-1}$ should be examined again (line 1 to 19). A PSR is build with each $M_i$ from the or sequences $OR_{s-1}$ in $S = (OR_1, OR_2, ..., OR_n)$. The algorithm also checks for each $M_i$ (line 5) if it contains a single category or a sequence of categories. In the first case, it finds the kth nearest neighbor $r_{|PSR|-1}$ in $U_{M_{i}}$ as the PNE algorithm would (line 6), otherwise it finds the k-th nearest neighbor to $r_{|PSR|-1}$ in $U_{M_i[1]}$ and initializes the list of categories $PSR.categories$ for the $PSR$ (line 10 to 16). From all these PSR the shortest one is chosen and put on the heap. 
If the indicator $PSR.prevPosition$ for the PSR is set ot $false$  it means that the k-th nearest neighbor to $r_{|PSR|}$ in $U_{PSR.categories[1]}$ is found (line 20 to 25). The indicator $PSR.prevPosition$ for the PSR is also updated to indicate that in \modifyRouteB{$PSR$} the k-th neighbor to the second to last PoI should be found. \newline

\begin{procedure}[H]
	\label{proc:modifyRouteB}
	\caption{modifyRouteB($PSR$)}
	\eIf{$PSR.prevPosition$}{
		\tcp{We check the OR sequence of the previous position $OR_{s-1}$}
		\ForEach(){$M_i$ in $OR_{s-1}$}{
			$initialize$ $shortestPSR$\;
			\tcp{Building a new $PSR$ with $M_i$}
			\eIf{$|M_i| = 1$}{
				\tcp{$M_i$ only contains one category}
				\KNN{$r_{|PSR|-1}, U_{M_{i}}$}\;
				$update$ $shortestPSR$\;
			}
			{
				\tcp{$M_i$ is a sequence of categories}
				\tcp{The neighbor from the first category $M_i[1]$ is found and the rest of the categries from $M_i$ are put into $categories$ for the specific PSR}
				\KNN{$r_{|PSR|-1}, U_{M_i[1]}$}\;
				$PSR.categories \leftarrow M_i[2..l]$\;
				$update$ $shortestPSR$\;
			}	
			
		}
		$newPSR \leftarrow shortestPSR$\;
	}{
		\tcp{We find the k-th neighbor to the second to last PoI}
		\KNN{$r_{|PSR|-1}, U_{M_{|PSR|}}$}\;
		$update$ $newPSR$\;
	}
	
	
	place $newPSR$ on heap\;
\end{procedure}


\subsection{Baseline approach} 
\label{sec:baselineOr}
The baseline approach on the or operator is entirely based on PNE by simply running PNE on all permutations of the query.

% Proving the correctness and comparing to the baseline/trivial approach
\subsubsection{Correctness}
\label{sec:correctnessOr}
\todo[color=yellow!40]{Todo: Correctness}

