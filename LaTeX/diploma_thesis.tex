% Contributors:

% Jan Stoess <stoess@ira.uka.de>
% Simon Kellner <kellner@kit.edu>
% Konrad Miller <miller@kit.edu>
%
\documentclass[12pt, a4paper]{book}
\usepackage{graphicx}
\usepackage{chngpage}
\usepackage{xspace,ifthen,epsfig}
\usepackage{cite}
\usepackage{color}
\usepackage{fancybox}
%\usepackage{pdfpages}
\usepackage{setspace}
\usepackage{subfigure}
\usepackage{longtable} 
\usepackage{tabularx} 
\usepackage{ltxtable} 
\usepackage{times}
\usepackage{url}
\usepackage{listings}
\usepackage{amsmath}
\usepackage[american,ngerman]{babel}
\usepackage[ansinew]{inputenc}
\usepackage{fancyhdr}
\usepackage{styles/kitthesiscover}
\usepackage[%dvipdfm,
   pdfauthor={Violina Zhekova},
   pdftitle={Flexible User-Friendly Trip Planning Queries},
   pdfsubject={B.sc. Thesis},
   pdfkeywords={Trip Planning}
]{hyperref}

\bibliographystyle{styles/plain}

%\raggedbottom
\newcommand{\todo}[1]{{\texttt{[#1]}}}
\newcommand{\code}[1]{{\tt \small{#1}}}
\newcommand{\evenindent}[2]{\ifodd #1 \else \hspace*{#2} \fi}

\begin{document}
\frontmatter
\unitlength1cm

\selectlanguage{american}


%% Titelseite

\title{Titel}
\author{cand. inform. Max Mustermann}
\thesistype{sa}
\primaryreviewer{Prof.\ Dr.\ Frank Bellosa}
%\secondaryreviewer{Prof.\ Dr.\ }
\advisor{Dipl.-Inform.\ Donald Duck}{}
\thesisbegindate{12.\ April 2010}
\thesisenddate{15.\ Dezember 2010}

\maketitle

\begin{otherlanguage}{ngerman}  
\thispagestyle{empty}
\vspace*{30\baselineskip}
\hbox to \textwidth{\hrulefill}
\par
\noindent Ich versichere wahrheitsgem"a"s, die Arbeit selbstst"andig verfasst, alle benutzten Hilfsmittel vollst"andig und genau angegeben und alles kenntlich gemacht zu haben, was aus Arbeiten anderer unver"andert oder mit Ab"anderungen entnommen wurde sowie die Satzung des KIT zur Sicherung guter wissenschaftlicher Praxis in der jeweils g"ultigen Fassung beachtet zu haben.\\

\noindent Karlsruhe, den \today
\end{otherlanguage}

%%%%%%%%%%%%%%%%%%%%%%%%%%%%%%%%%%%%%%%%%%%%%%%%%%%%%%%%%%%%%%%%%%%%%%%%
%% Hinweis:
%%
%% Diese Erklärung wird von der Prüfungsordnung für Diplomarbeiten 
%% verlangt und ist zu unterschreiben. Für Studienarbeiten ist diese
%% Erklärung nicht zwingend notwendig, schadet aber auch nicht.
%%%%%%%%%%%%%%%%%%%%%%%%%%%%%%%%%%%%%%%%%%%%%%%%%%%%%%%%%%%%%%%%%%%%%%%%
\clearpage








\chapter{Abstract}

Trip planning queries are often from the type Sequenced route Queries (SRQ), a form of nearest neighbor queries, which define a starting point and a list of categories, given by the user. This type of queries are gaining significant interest, because of advances in location based mobile services and they are also of great importance in developing robust systems, where crisis management is of utter importance. 

\noindent Existing approaches strive to find a best route, based on length, duration or other prime factors, passing through multiple location, called points of interest (PoIs), and they match the route perfectly. However, users may be also interested in other qualities of the route, such as the relationship among sequence points, hierarchy, order and priority of the PoIs. Therefor, in this thesis  I introduce a couple of operators, which the users may be interested in applying to SRQ, and propose approaches to designing and implementing some of the operators. The implementation considers only metric spaces, as these are mostly relevant to the user, when working with road networks in real-life maps.

\mainmatter
\cleardoublepage
\phantomsection
\addcontentsline{toc}{chapter}{Contents}
\tableofcontents 

\chapter{Introduction}
\label{sec:intro}
A sequenced route query is defined as finding the shortest path from a starting point towards a possible destination, passing through multiple locations, defined by their category type. There has been significant research and proposed approaches on the topic, but there is not a developed query language to answer this types of queries. The work in this thesis has been focused on researching the topic of sequenced route queries and designing a language to enable the user to express his need in the form of a user query in a flexible manner, such as applying different constraints on the route to be found. 

\textbf{Example:}
Suppose that a user is planning a trip to town: he first wants to go to a restaurant for lunch, then he wants to stop by a bank, then he meets a friend in the shopping mall and after that he plans to have a dinner at a restaurant. In this specific scenario, the user wants to express his wish for the restaurant to be the same.

The specific constrains such as the equality in the given example above are proposed in the thesis as operators on the query. Existing approaches have been used to transform the complex user query and changes to the approaches have been made in order to retrieve a desired result. 

The completeness of the operators stems from %should put grafic (? extensive research on the topic).

\textbf{Problem definition:}
We have a starting point $sp$ and a category sequence $M = (c_1, c_2, ..., c_n)$, which constitutes the query, defined by the user. The constraints for this query can be applied as operators.
For this query a route $(r_1, r_2, ..., r_n)$, defined as a sequence of PoIs, is calculated.

\textbf{Construction:}
The graph is constructed using the Neo4j graph database as seen in %should put grafic
The crossroads are defined as nodes, labeled CROSSROAD, with attributes id, longitute and latitude, the roads are mapped as relationships, labeled ROAD, with attribute distance. A PoI is mapped to the nearest CROSSROAD, using a relationship HAS-POI, with attributes id and a list of possible category types it belongs to. 
The map used for implementation and testing is the road network of Berlin, with 428769 nodes, 504229 roads, 5548 PoIs and 7 category types: restaurant, coffee shop, atms/banks, movie theaters, pharmacies, pubs/bars, gas stations. 
\newline

The remainder of the thesis is organized as follows: I first review the related work that has been done on the topic of SRQ in Section 2. In Section 3 I cover the proposed operators and I go into details on some of them in three separate sections for each of them: Design, Implementation and Evaluation. Finally, I conclude the thesis by summing up the progress made on the subject and discuss future work.

\cite{mach87ulvm}

\chapter{Related Work}
\label{sec:bgrelwork}

\chapter{Design} 
\label{sec:design}

\chapter{Implementation}
\label{sec:implementation}

\chapter{Evaluation}
\label{sec:evaluation}

\chapter{Conclusion}
\label{sec:conclusion}

\backmatter    
%\chapter{Bibliography}
\phantomsection
\addcontentsline{toc}{chapter}{Bibliography}
\bibliography{diploma_thesis}
\end{document}
