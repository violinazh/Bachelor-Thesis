\chapter{Related Work}
\label{sec:relwork}
In this section I would like to review some existing research, related to the topic of this thesis. Sequenced route queries have been extensively researched and different algorithms that optimize the problem and address different use scenarios have been developed. Usually, existing approaches differentiate between vector and metric spaces, considering the Euclidean distance between geographic points or the real-life road-network-based distances accordingly. Some algorithms are focused on returning a single optimal route, where the PoIs match the given categories in the category sequence perfectly, whereas others consider semantic hierarchy or multiple route factors such as rating, distance and category weights. 

In \textit{The Optimal Sequenced Route} the researchers propose two effective algorithms for solving the sequenced route query problem. They first elaborate on why a classic shortest path algorithm such as Dijkstra would be impractical for real-life scenarios and then go on to propose the LORD (Light Optimal Route Discoverer) and R-LORD algorithm, which uses a R-tree, which are Dijkstra-based and made for vector spaces and the PNE (Progressive Neighbor Exploration) algorithm, which emplys the nearest neighbour search and is designed specifically for metric spaces. Both of their proposed algorithms calculate a perfect route and only return one optimal route (while modification of the PNE algorithm also allow for finding k optimal routes), significantly outperforming Dijkstra's algorithm. \cite{OSR}

A different approach to the SRQ, designed for metric spaces, is proposed in \textit{Sequenced Route Query with Semantic Hierarchy}. The authors suggest a Skyline based algorithm, called bulk SkySR (BSSR), which searches for all preferred routes to users by extending the shortest route search with the semantic similarity of PoIs' categories. This approach expects a category tree, representing the semantic hierarchy of categories, and applies the Skyline concept, which is searching for routes that are not worse than any other routes in terms of their scores, to the route length and semantic similarity, also known as the route scores. The BSSR algorithm also exploits the branch-and-bound concept by searching for routes simultaneously to reduce the search space. \cite{semanticSRQ}

Another research article proposes the Personalized and Sequenced Route (PSR) Query, which considers both personalization and sequenced constraints. The approach takes into account multiple factors of a route, such as distance rating and associates different weight with each PoI category and a distance weight. The framework designed to obtain one optimal route consists of three phases: guessing, crossover and refinement, and is focused on spatial databases. \cite{personalSRQ} 

In \textit{In-Route Skyline Querying for Location-Based Services} queries are issued by user moving along a routes towards destinations (PoIs), also defined as query points. The movement of the user is constrained to a road network and the travel distance is considered. In-route queries know the destination and current location of the user, which dynamically changes, and the anticipated route towards the endpoint. Users can apply weights to several spatially-related criteria, when deciding on PoIs to visit next, such as the total distance difference, known as detour, and the relative distance of the current data point. \cite{dynamicSRQ}

An article \textit{Sequenced Route Queries: Getting Things Done on the Way Back Home} suggest speedup techniques for sequenced route queries. A contraction hierarchy is proposed for preprocessing results for faster retrieval of answers by shortest path queries in road networks. The second technique uses the distance sensitivity of routes ("most queries are of a local kind"), which it bases on users' typical behavior. In this approach, one optimal route is returned, but queries where the order of PoIs is not necessarily fixed are possible as long as the number of PoIs remains moderate. Also, constraints on the order of visited PoIs can be made, e.g. visiting a restaurant before a shopping center. \cite{skyline}