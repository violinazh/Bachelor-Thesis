\chapter{Related Work}
\label{sec:relwork}
In this section we would like to review some existing research, related to the topic of this thesis. Route queries have been extensively researched and various algorithms that optimize the problem and address different use scenarios have been developed. Usually, existing approaches differentiate between vector and metric spaces, considering either the Euclidean distance between geographic points or the real-life road network distances. Some algorithms are focused on obtaining a single optimal route, where the PoIs match the given CoIs in the category sequence perfectly, whereas others consider semantic hierarchy or multiple route factors such as rating, distance and category weights and retrieve multiple possible routes. There is also the distinction between Trip Planning Queries (TPQ) \cite{tpq} and Optimal Sequenced Route (OSR) Queries  \cite{OSR}. While TPQ searches for the optimal route from a starting point towards a destination, passing through a set of CoIs without specified order, the OSR query has a starting point and a predefined order (sequence) on the CoIs, which are to be visited. For the obtaining of optimal route, different route parameters may be considered such as route length, travel duration, ranking of the PoIs or dynamic traffic information.

In \textit{The Optimal Sequenced Route} \cite{OSR} the researchers propose two effective algorithms for solving the Optimal Sequenced Route (OSR) query. They first elaborate on why a classic shortest path algorithm such as Dijkstra would be impractical for real-life scenarios and then go on to propose the LORD (Light Optimal Route Discoverer) and R-LORD algorithm, which uses an R-tree structure, for vector spaces and the PNE (Progressive Neighbor Exploration) algorithm, which employs the nearest neighbor search and is designed specifically for metric spaces. Both of their proposed algorithms calculate a perfect route and return one optimal route (while modification of the PNE algorithm also allow for finding k optimal routes), significantly outperforming Dijkstra's algorithm. The authors of \textit{Processing Optimal Sequenced Route Queries Using Voronoi Diagrams} \cite{voronoi} extended this research and proposed a pre-computation approach in both vector and metric spaces to answer the OSR query by taking advantage of a family of AW(additively weighted)-Voronoi diagrams for different CoIs sets. The approach ultimately outperforms the previous index-based approaches in terms of query response time.

Another solution to solving the OSR query with included destination point is demonstrated in \textit{Sequenced Route Query in Road Network Distance Based on Incremental Euclidean Restriction} \cite{betterOSR}, where a framework based on the incremental Euclidean restriction (IER) approach is employed, which searches candidates in the Euclidean distance first, and then verifies the results of the road network distance. Compared to PNE, the approach performs better with densely distributed CoI sets or when the number of CoIs to be visited during the trip is large.

A different approach to the SRQ, designed for metric spaces, is proposed in \textit{Sequenced Route Query with Semantic Hierarchy} \cite{semanticSRQ}. The authors suggest a Skyline based algorithm, called bulk SkySR (BSSR), which searches for all possible optimal routes by extending the shortest route search with the semantic similarity of CoIs. This approach expects a category tree, representing the semantic hierarchy of categories, and applies the Skyline concept, proposed in \cite{skyOp}, which is based on searching for routes that are not worse than any other routes in terms of their scores - route length and semantic similarity. The BSSR algorithm also exploits the branch-and-bound concept by searching for routes simultaneously to reduce the search space. 

Another research work proposes the Personalized and Sequenced Route Query \cite{personalSRQ}, which considers both personalized and sequenced constraints. The approach takes into account multiple factors of a route, such as distance rating and associates different weight with each CoI and a distance weight. The framework designed to obtain one optimal route consists of three phases: guessing, crossover and refinement, and is focused on spatial databases. 

In \textit{In-Route Skyline Querying for Location-Based Services} \cite{dynamicSRQ} queries are issued by the user moving along a routes towards destinations (PoIs), also defined as query points. The movement of the user is constrained to a road network and the travel distance is considered. In-route queries know the destination and current location of the user, which dynamically changes, and the anticipated route towards the endpoint. Users can apply weights to several spatially-related criteria, when deciding on PoIs to visit next, such as the total distance difference, known as detour, and the relative distance of the current data point. However, this approach issues continuous skyline queries in search for a single PoI, and it is therefore not applicable to SkySR queries, which obtain routes that pass through multiple PoIs at a time. 

Furthermore, \textit{A Continuous Query System for Dynamic Route Planning} \cite{dynamic} researches the problem of answering continuous route planning queries in a road network environment, while taking the delay of updates into account, and aim to obtain the shortest path.

An attempt to tackle interactive route search is presented in \textit{An Interactive Approach to Route Search} \cite{interactive}. The user can move from a starting location towards a destination, while making search queries for points of different types on the way. Upon arrival at venues, the user gives feedback specifying whether the venue satisfies its corresponding query. The proposed heuristic algorithms compute the next object to be visited, based on the feedback.

Another article \textit{Sequenced Route Queries: Getting Things Done on the Way Back Home} \cite{skyline} suggests speedup techniques for sequenced route queries. A contraction hierarchy is proposed for preprocessing results for faster retrieval of answers by shortest path queries in road networks. The second technique uses the distance sensitivity of routes ("most queries are of a local kind"), which it bases on users' typical behavior. In this approach, one optimal route is returned, but queries where the order of CoIs is not necessarily fixed are possible as long as the number of CoIs remains moderate. Also, constraints on the order of visited CoIs can be made, e.g. visiting a restaurant before a shopping center. 

The authors of \textit{On Trip Planning Queries in Spatial Databases} \cite{tpq} proposed approximation algorithms for Trip Planning Queries (TPQ) in a metric space. With TPQ, the user specifies a set of CoIs and asks for the optimal route from her starting location to a specified destination which passes through exactly one PoI in each CoI. In comparison to an OSR query, there is no order imposed on the types of POIs to be visited in a TPQ query. A multi-type nearest neighbor (MTNN) query solution was proposed in \textit{Exploiting a Page-Level Upper Bound for Multi-Type Nearest Neighbor Queries} \cite{mtnn}. Given a starting point and a set of categories, a MTNN query finds the shortest path for the query point such that only one instance of each category is visited during the trip. MTNN is compared against RLORD, proposed in \cite{OSR}, which was designed to solve the spatially constrained OSR. Thus MTNN can be seen as an extended solution of OSR, which uses an R-Tree structure and exploits a page-level upper bound (PLUB) for efficient pruning at the Rtree node level. The approach is only superior to RLORD, when datasets are compact.

\textit{The partial sequenced route query with traveling rules in road networks} \cite{multi} handled the multi-rule partial sequenced route (MRPSR) query, which enables the user to efficiently plan a trip by defining a number of traveling rules. The MRPSR query provides a unified framework that subsumes the trip planning query (TPQ) and the optimal sequenced route (OSR) query. It proposes three algorithms (Nearest Neighbor-based Partial Sequence Route query (NNPSR) algorithm, NNPSR combined with the Light Optimal Route Discoverer (LORD) algorithm, Advanced A* Search-based Partial Sequence Route query (AASPSR(k))) for solving the problem in road networks with a sub-optimal solution. The paper \textit{Optimal Route Queries with Arbitrary Order Constraints} \cite{optMulti}, on the other hand, focuses on efficient, exact methods for the optimal TPQ, while also giving the option for visiting order constraints. The proposed approach uses backward search and forward search and proposes four algorithms (Simple Backward Search (SBS), Batch Backward Search (BBS), Simple Forward Search (SFS), Batch Forward Search (BFS)), whereas the BFS algorithm that combines merits from both backward and forward search achieves the best performance. However, both approaches are based on the Euclidean distance between PoIs and therefore could be difficultly applied to metric spaces.

A linked area of research studies min-cost path queries. In \textit{Monitoring Minimum Cost Paths on Road Networks} \cite{minCost} the continuous min-cost path query is proposed and a system, PathMon, is presented to monitor min-cost routes in dynamic road networks, attempting to capture the characteristics of ever-changing road networks. Another paper \cite{spatialDB} deals with query processing in spatial network databases and proposes an architecture that integrates network and Euclidean information, capturing pragmatic constraints.It designs algorithms based on the Euclidean restriction and a network expansion frameworks that take advantage of location and connectivity to efficiently prune the search space.

However, all of the aforementioned solutions have not yet attempted to design a uniform route query language for the road network's metric space with the ability to provide the user with options to apply different rules to the query and to retrieve an optimal result in the end. This will be the focus of this thesis.