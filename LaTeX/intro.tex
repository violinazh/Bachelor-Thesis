\chapter{Introduction}
\label{sec:intro}

\section{Motivation}
\label{sec:motivation}
A sequenced route query is defined as finding the shortest path from a starting point towards a possible destination, passing through multiple locations, defined by their category type. There has been significant research and proposed approaches on the topic, but there is not a developed query language to answer this types of queries. The work in this thesis has been focused on researching the topic of sequenced route queries and designing a language to enable the user to express his need in the form of a user query in a flexible manner, such as applying different constraints on the route to be found.

\textbf{Example:}

Figure \ref{fig:example} presents a small example network with four different types of point sets, illustrated with different colors. The colored circles represent the PoIs and the lines between them represent the roads, on each road there is a label with its length. There is two restaurants, a movie theater, two banks and two pharmacies. The starting points \textit{sp} is represented by a rhombus.

Suppose that a user is planning a trip to town: he first wants to go to a restaurant for lunch, then he wants to stop by a bank, then he meets a friend in the movie theater and after that he plans to have a dinner at a restaurant. In this specific scenario, the user wants to express his wish for the restaurant to be the same, because he may prefer a route where the equality of the two restaurant PoIs is more important to him than the length of the route.

With existing approaches, the user can get the shortest route \cite{OSR} or all routes that satisfy the semantic similarity and length conditions equally \cite{semanticSRQ}, but that does not guarantee the equality of the two restaurant PoIs. Also finding k optimal routes answering the user's SRQ and then filtering out the routes where the two PoIs of type restaurant are equal has proven to not always generate a result, which is why in this thesis a better optimal approach is presented.

Using Figure \ref{fig:example}, we see that the user query in the example above, when answered as an OSR query, delivers the route $(r_1, b_1, mt_1, r_2)$ with length 11 (shown with red lines in the figure), where both restaurants $r_1$ and $r_2$ are different. Our approach strives to find the optimal route, where the two restaurants are equal. It will not necessarily be shortest possible sequenced route, but it will be the shortest route out of all possible routes, where the two PoIs are equal to each other. Such route in our example graph would be $(r_1, b_1, mt_1, r_1)$ with length 12 (shown with dashed lines in the figure).

Specific constrains such as the equality in this example are proposed in the thesis as operators on the query. We modified and extended existing approaches, which answer the Optimal Sequenced Route (OSR) Query such as the Progressive Neighbor Exploration (PNE) \cite{OSR} in order to transform the complex user query and retrieve a desired result.

\begin{figure}[h]
	\includegraphics[scale=1]{images/Example_routes.png}
	\centering
	\caption{A network with four different types of point sets}
	\label{fig:example}
\end{figure}

\section{Problem definition}

We have a starting point and a category sequence, which constitutes the SRQ. In addition to this two query parameters, the user can define other parameters, which are applied to the route query as query operators. Such possible query operators, presented in this thesis, are the equality operator, not-equality operator, or operator and order operator. 

Then we define the \textit{Complex Route Query (CRQ)}:

\textbf{Complex route query (CRQ):} Given a starting point $sp$, a category sequence $M = (c_1, c_2, ..., c_n)$ and an operator $OPERATOR$, $Q(sp, M, OPERATOR)$ is a Complex Route (CR) Query, which searches for the optimal route $R = (r_1', r_2', ..., r_l')$, defined as a sequence of PoIs.
%Depending on which operator is applied to the query the found route $R$ follows $M$ in a way that the operator states

In Chapter \ref{sec:operators} the definitions of the separate operators are presented individually.

% e.g. performance, processing time for the trivial approach 
\section{Challenges}
\todo[color=yellow!40]{Todo: Challenges}

% Shortly presenting the algorithms for the operators; experiment results, comparing to the baseline approach
\section{Contributions}
\todo[color=yellow!40]{Todo: Contributions}

\section{Outline}
The remainder of the thesis is organized as follows: First, I review the related work that has been done on the topic of SRQ in Section 2. In Section 3 I cover the proposed operators and go into details on some of them in three separate sections for each of them: Design, Implementation and Evaluation. Finally, I conclude the thesis by summing up the progress made on the subject and discuss future work.