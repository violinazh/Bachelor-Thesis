% Introduce the purpose, short summary of the thesis and main conclusion
\chapter{Abstract}
\enlargethispage*{30pt}

Trip planning queries are often from the type Sequenced Route Queries (SRQ), a form of nearest neighbor queries, which define a starting point and a list of categories to be visited in a consecutive order, given by the user. This type of queries are gaining significant interest, because of advances in Location Based Services (LBS) and they are also of great importance in developing robust Geographic Information System (GIS) applications (e.g. logistics and supply chain management), where crisis management is of utter importance. 

Existing approaches strive to find a best route, based on length, duration or other prime factors, passing through multiple location, called Points of Interest (PoIs), and they match the route perfectly. However, users may be also interested in other qualities of the route, such as the relationship among sequence points, hierarchy, order and priority of the PoIs. Therefore, in this thesis we introduce a set of operators, which the users may be interested in applying to SRQ, and propose approaches to designing and implementing some of the possible operators. The implementation considers metric spaces, as these are mostly relevant to the user, when working with road networks in real-life maps.
The query operators this thesis has focused on are the following: "equality" operator (EO), "inequality" operator (NEO), "or" operator (OR) and "order" operator (ORDER). The "equality" operator gives the user the flexibility to build a query, where two PoIs of the same category must be equal. The inequality answers the opposite problem. The "or" operator provides the user with the opportunity to give multiple variants of a SRQ, using or operands. With the "order" operator the user is able to assign fixed positions to one or more categories in the route query, whereas it is then no longer viewed as a sequenced route query and should thus be answered differently than a simple SRQ.

The approaches used to implement the query operators include the Progressive Neighbor Exploration (PNE) approach, presented in \cite{OSR} and heuristic methods for shrinking the search space. The extensive experiments, which were done using a real-world dataset verified that the created algorithms perform as expected and outperform the baseline approaches in terms of processing time and required workspace.

\pagebreak