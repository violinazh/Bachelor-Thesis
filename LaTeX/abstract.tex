% Introduce the purpose, short summary of the thesis and main conclusion
\chapter{Abstract}

Trip planning queries are often from the type Sequenced route Queries (SRQ), a form of nearest neighbor queries, which define a starting point and a list of categories, given by the user. This type of queries are gaining significant interest, because of advances in location based mobile services and they are also of great importance in developing robust systems, where crisis management is of utter importance. \newline

Existing approaches strive to find a best route, based on length, duration or other prime factors, passing through multiple location, called points of interest (PoIs), and they match the route perfectly. However, users may be also interested in other qualities of the route, such as the relationship among sequence points, hierarchy, order and priority of the PoIs. Therefore, in this thesis  I introduce a set of operators, which the users may be interested in applying to SRQ, and propose approaches to designing and implementing some of the operators. The implementation considers metric spaces, as these are mostly relevant to the user, when working with road networks in real-life maps.

\todo[color=yellow!40]{main conclusion}