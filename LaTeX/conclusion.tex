% Summing up, unresolved problems and further research
\chapter{Conclusion and Future work}
\label{sec:conclusion}
\enlargethispage*{30pt}

We studied the different types of route queries, that are currently present in the research, and we identified the lack of a query language designed for the user's specific needs for flexibility when it comes to route queries. In order to attempt to fill this gap, we proposed four operators, which are specifically designed for route queries in metric spaces and the prime factor upon which routes are quantified is travel distance. 
We first presented the equality operator, which was designed based on the PNE approach \cite{OSR} and a heuristic to reduce the search space. We showed through evaluation experiments that the proposed approach outperforms the baseline approach in all of the experiment parameters. Next, we introduced the not-equality operator as extension to the equality operator. For this operator we modified the PNE algorithm in order to get the desired result. Furthermore, we presented the or operator, which strives to give the user maximum flexibility when it comes to building a multiple-option route query. For this operator we developed an algorithm to work with multiple query options simultaneously and when we compared it with the baseline approach, it proved to perform significantly better, especially when many or operands are presents. Last but not least, we designed the order operator, which gives the user the ability to construct partially or even fully not sequenced route query. This approach was also designed similarly to the or operator by developing a method to handle and compare multiple query options at once. When comparing with the baseline approach, we concluded that the proposed approach performs exponentially better than the trivial solution whenever the number of ordered elements decreases.

With these finding in mind, we consider our work to has successfully delivered four query language operators that could be implemented and used in location based services. In comparison with other attempts at solving the multi-rule route queries \cite{multi}, our proposed approaches deliver an optimal result. It is important to note, however, that the mentioned operators are definitely not a complete set of query language operators, but they represent the four ones, that have proved to be the most desired in query language overall and therefore fulfill the user requirements the most. Nevertheless, while researching the topic of route queries, we came up with multiple ideas for operators that could be further researched and implemented. Such operator is for example the hops operator, with which the user can explicitly define how many location visits he wants to make between two categories in the category sequence. This operator is very similar to the order operator and could be designed similarly. Furthermore, an operator could be developed, with the help of which the user can specify necessary categories, which must be present in the final result. Considering this, the query route would only contain the PoIs of unnecessary CoIs, if these are situated on the way between the necessary points and the route's length does not increase with by adding them. Although still useful as a separate operator, the user could also achieve the same with our already developed or operator. \newline
Another ideas for operators, which are targeted more towards the semantic hierarchy of the categories in the user-specified category sequence, could be the and and not operators. These two operators could be applied to specific categories in the user query's category sequence and could be useful in the case that a single PoI has many category types. For example of the and operator, a user may want to go to a cafe, that is also a restaurant, or in the case of the not operator, he may want to visit a restaurant, which is also specifically not categorized as a coffee shop. \newline
All of these mentioned operators could be also inspired by the PNE approach that we used for the four operator in this thesis. But another solution could be to expand on the \textit{Skyline Sequenced Route Query} (SkySR) \cite{skyline}. This method searches for routes based on the Skyline concept, which entails finding routes that are not worse than any other routes in terms of theirs scores - route length and semantic similarity. This could be another alternative to give more flexibility to the user by making use of the semantic similarity of CoIs. An operator that could be useful in this case is the perfection operator, applied to categories in the category sequence. With this operator, a user can specifically force the Skyline algorithm to match the category to which the operator has been applied perfectly in terms of semantic similarity. Finally, there is probably many other operators that would come up with further scientific work and research on the topic of route queries and the ones we summarize here are only our suggestions to the topic. For future work, it would be also interesting to extend the proposed algorithms to also support dynamic road networks, in which traffic information is provided (e.g., travel time, traffic congestion, etc.).

In conclusion, we think that with the development of four operators for the route query language we have successfully contributed to this field of research. We have also managed to identify linked topics for further study and to make recommendations on how to expand the query language. 

\pagebreak
