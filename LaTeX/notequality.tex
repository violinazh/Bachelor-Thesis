\SetKw{Return}{return}
\SetKw{Break}{break}

\SetKwFunction{PNE}{PNE}
\SetKwFunction{NN}{NN}
\SetKwFunction{KNN}{kNN}
\SetKwFunction{travelDistance}{travelDistance}
\SetKwFunction{length}{length}
\SetKwFunction{trim}{trim}

\SetKwBlock{A}{a)}{end}
\SetKwBlock{B}{b)}{end}

\section{"Inequality" operator}

\subsection{Motivation}
\label{sec:motNEO}
The "inequality" operator is based on the need to express that some PoIs in the SRQ of the same category should not be equal. Following a similar example ot the one in Section \ref{sec:motivation} for the road network in Figure \ref{fig:example}, a user is planning a trip to town: he first wants to go to a restaurant for lunch, then he wants to stop by a pharmacy and after that he wants to go to a different restaurant for coffee. In this specific scenario, the user wants to express his wish for the restaurants to be the different, but with the existing PNE \cite{OSR} approach, which finds an OSR, the user would get the route $(r_1, ph_1, r_1)$ with length 5, where both restaurants are the same. The proposed approach in this chapter strives to find the optimal route, where the two restaurants are different. It will not necessarily be the shortest possible sequenced route, but it will be the shortest route out of all possible routes, where the two PoIs are different. Such route in our example graph would be $(r_1, ph_2, r_2)$ with length 7 (shown with dashed lines in the figure).

\subsection{Problem definition} 
\label{sec:problemNEO}
The "inequality" operator is defined as follows:

\textbf{"Inequality" operator:} Given a sequence of categories $M = (c_1, c_2, ..., c_l)$, a starting point $sp$ in ${\rm I\!R}^2$ and indices $i$ and $j$, , where $r_i \in C_{M_{i}}$, $r_j \in C_{M_{j}}$ and $M_i \neq M_j$, $UNEQUAL(i, j)$ is an "inequality" operator, which states that $r_i$ and $r_j$ in the found route $R = (r_1, r_2, ..., r_l)$ should be different points of interest.
$Q(sp, M, UNEQUAL(i, j))$ is a Sequenced Route (SR) Query, which searches for the shortest (in terms of function $length$) sequenced route $R$ that follows $M$ and where $r_i \neq r_j$.

%\subsection{Precomputations} 
%\label{sec:precompNEO}
%The precomputations are the same as for the not equality operator in Section \ref{sec:precompEO}.

% Introducing the proposed algorithm + proving the correctness
\subsection{Proposed approach} 
\label{sec:approachNEO}
The "inequality" operator is designed using the PNE approach, proposed in \cite{OSR}. It uses the progressive neighbour explorator as its base to upgrade on and explore all the possible optimal routes until it finds an optimal route, in which the given PoIs are different.

% Explaining the algorithm step by step
\subsubsection{Algorithm}
\label{sec:algortihmNEO}
The algorithm \texttt{\nameref{alg:notequality}} starts by initializing the $heap$, ordered by the length of the routes, and the first PSR (line 1, 2, 3). It then proceeds to inspect and modify the routes on the $heap$ based on their length, until a full SR is found (line 7 to 20). Each full SR (line 8 to 11) is checked if it is a full SR. If this is the case, the found optimal SR is returned, otherwise the next PSR on the $heap$ is fetched. If the fetched PSR is not a full SR but a partial route (line 12 to 19), then the algorithm performs a) and b) as in the original PNE algorithm. \newline

\begin{algorithm}[H]
\caption{notEqualityOperator()}
\label{alg:notequality}
	
	\SetKwInOut{Input}{Input}\SetKwInOut{Output}{Output}
	
	\Input{$(sp, M = (c_1, c_2, ..., c_l)), UNEQUAL(i, j)$}
	\Output{$R = (r_1, r_2, ..., r_l)$}
	\BlankLine
	
	$initialize$ $heap$\;
	$initialize$ $candidate$\;
	
	$firstPSR =$\NN{$sp, C_{M_{1}}$}\;
	place $firstPSR$ on $heap$\;
	
	\BlankLine
	
	\tcp{At each iteration of PNE a $PSR$ (partial sequenced route) is fetched and examined based on its length and it is checked}
	
	fetch a $PSR$ from the $heap$\;
	\Switch{$s = size(PSR)$}{
		\Case{$s == l$}{
			$PSR$ is the optimal route\;
			\Return $PSR$\;
		}
		\Case{$s \neq l$}{
			\A{
				\NN{$r_{|PSR|}, C_{M_{|PSR|+1}}$}\;
				update $PSR$\;
				\trim{$PSR$}\;
			}
			\B{
				\KNN{$r_{|PSR|-1}, C_{M_{|PSR|}}$}\;
				generate a new $PSR$\;
				\trim{$PSR$}\;
			} 
			
		}
	}
	
\end{algorithm}

\pagebreak

Each time a $PSR$ is generated the procedure \texttt{\ref{proc:trim_NEO}()} is performed. It is checked if the generated $PSR$ is a full SR (line 1) and if that is the case (line 1 to 16) the route is further examined. In the case that the route satisfies the requirement for $i$ and $j$ to not be equal (line 2 to 8), its length is compared to that of the candidate route and if it is shorter or equal, the candidate route is updated and the sequenced route is placed on the $heap$. Otherwise (line 8 to 15) we check if $j$ is the last index in the route and in this case (line 10 to 14) the k-th neighbor of the previous PoI to the last one is found and a new $PSR$ is generated. If the route does not satisfy the requirements for $i$ and $j$ to not be equal and $j$ to be last index, it is simply discarded as a not possible SR. In case that the generated route is still not a full SR (line 16 to 21) the $PSR$ is placed on the $heap$. \newline

\begin{procedure}[H]
\caption{trim($PSR$)}
\label{proc:trim_NEO}
	
	\eIf{$size(PSR) = l$}{
		\eIf{$PSR[i] \neq PSR[j]$}{
			\tcp{Optimization: length check}
			\If{$length(PSR) <= length(candidate)$}{
				update $candidate$\;
				place $PSR$ on the $heap$\;
			}
		}
		{
			\tcp{In case $j$ is the last index in the route, we find the kth neighbor of the previous PoI to the last one}
			\If{$size(PSR) = j + 1$}{
				\KNN{$r_{|PSR|-1}, C_{M_{|PSR|}}$}\;
				generate a new $PSR$\;
				\trim{$PSR$}\;
			}
		}
		
	}{
		\tcp{Optimization: length check}
		\If{$length(PSR) <= length(candidate)$}{
			place $PSR$ on the $heap$\;
		}
	}
\end{procedure}

\subsubsection{Running example}
We describe the algorithm for the "inequality" operator using the example in Section \ref{sec:motNEO}. The user in the example wanted to visit a restaurant, a pharmacy and a restaurant again, but he wanted both restaurants to not be equal ($M = (r, ph, r)$, $|M| = l = 3$, $UNEQUAL(0, 2)$). In Figure \ref{heapNEO} the partial routes stored in the $heap$ in each step of the algorithm are displayed.
First, the nearest neighbor to the starting point $sp$, $r_1$  is found and a new $firstPSR$ is generated and put on the $heap$: $(r_1)$ with length 1. In step 2 route $(r_1)$ is fetched from the $heap$ and modified according to a) and b) as in the PNE algorithm. Similarly, this process is repeated until the route on top of the $heap$ follows the sequence $M = (r, ph, r)$. the difference here from the PNE algorithm comes from the trimming part. For example in step 2, route $(r_1, ph_1)$ is fetched from the $heap$ and in a) the full SR $(r_1, ph_1, r_1)$ is generated. In the trimming function, it is checked if the restaurants at positions 0 and 2 are equal. In case they are not the candidate route is initialized and placed on the $heap$, but in our case, they are equal and $j$ is also the last index 2, which means that the next neighbor to $ph_1$ is found and the alternate route $(r_1, ph_1, r_2)$ with length 10 is generated and placed on the $heap$. Then some more PSR are generated, until in step 5 from $(r_1, ph_2)$ again a full SR $(r_1, ph_2, r_2)$ with length 7 is generated. The route is then trimmed and since the restaurants are not equal and also the route is shorter than the current candidate route, the candidate route is updated and the new SR is placed on the $heap$. In step 6 the fetched route $(r_1, ph_2, r_2)$ is a full SR, which satisfies the condition for the restaurant at indices 0 and 2 to not be equal, is returned as the optimal route.

\begin{table}[h]
	\centering
	\begin{tabular}{ |l|l| } 
		\hline
		Step & Heap contents (PSR $R : length(R)$) \\
		\hline
		1 & $(r_1 : 1)$ \\ 
		 \hline
		2 & $(r_1, ph_1 : 3), (r_2 : 5)$ \\ 
		\hline
		3 & $(r_2 : 5), (r_1, ph_2 : 5), (r_1, ph_1, r_2 : 10)$ \\ 
		\hline
		4 & $(r_1, ph_2 : 5), (r_2, ph_2 : 7), (r_1, ph_1, r_2 : 10)$ \\ 
		\hline
		5 & $(r_1, ph_2, r_2 : 7), (r_2, ph_2 : 7),$ \st{$(r_1, ph_1, r_2 : 10)$} \\ 
		\hline
	\end{tabular}
	\caption{Steps of the algorithm for NEO using the road network from Figure \ref{fig:example}}
	\label{heapNEO}
\end{table}

% Proving the correctness and comparing to the baseline/trivial approach
\subsubsection{Correctness}
\label{sec:correctnessNEO}
%??? OSR Property - last index
We need to prove that no suitable partial sequenced routes have been pruned by the proposed algorithm.

\textbf{Lemma 2:} For given query $Q(sp, M, UNEQUAL(i, j))$, the algorithm for the "inequality" operator returns an optimal route in terms of route length.

\textit{Proof by contradiction:} Let $R' = (r_1', r_2', ..., r_l')$ to be the optimal route and $R = (r_1, r_2, ..., r_l)$ to be the route that the algorithm finds. We assume $length(R') < length(R)$ and show that this is not possible. In order to do this, we differentiate between two separate cases: 

Case 1: At some point in time we have the partial routes $P = (r_1, ..., r_k, r_{k+1})$ and $P' = (r_1, ..., r_k, r'_{k+1})$, which differ only in their last PoI. Since the result route is $R$ at this point the algorithm must choose $P$ as the partial route to extend. This means that $P$ has a shorter length than $P'$, because the heap is ordered by the length of the route. Since the algorithm returns R as the optimal route it means that $length(P') > length(R)$. But because it also holds true that $length(R') > length(P')$, it implies $length(R') > length(R)$, which is a contradiction.

Case 2: At no point partial route $P'$ exists, which is to be extended to sequenced route $R'$. This means that the length of the partial route would exceed the length of the candidate route on top of the heap $length(P') > length(R)$ and is therefore never generated. This implies $length(R') > length(R)$ and this is a contradiction.

Lemma 2 shows that the algorithm for the "inequality" operator progressively searches the entire space of candidate sequenced routes with unequal PoI at indices $i$ and $j$ and returns the optimal route. Thus the correctness of the proposed approach is proved.